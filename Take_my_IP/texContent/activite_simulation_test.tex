\section{Simulations et tests d'intégration}

\subsection{Les tests d'intégration}

Il s'agit dans la sous-section suivante de présenter les différents tests d'intégrations.
La liste à puce ci-dessous présente les divers tests avec une brève présentation, les éléments validant la bonne intégration de l'IP et les besoins en terme d'outils et logiciels.

\begin{itemize}
	\item 
\end{itemize}

\subsection{Étapes non nécessaires}

À ce stade du développement de l'IP, certaines étapes étudiées ne sont pas nécessaires.\\

Par exemple, les étapes de modélisation et de vérification au niveau système réalisable sur SystemC.
Il n'y a aucun profit à tirer de cet outil car l'IP est déjà décrite et la spécification et conception au niveau système ont déjà été pensées.\\

En ce qui concerne l'outil de codesign Cofluent étudié en travaux pratiques, celui-ci est utile en amont de la conception.
Cofluent permet de réaliser des études de faisabilité, déterminer des propriétés pour un circuit donné et effectuer du design exploration.
