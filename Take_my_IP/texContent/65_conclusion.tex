\section{Conclusion}
Ce document présente le travail à mener afin d’effectuer l’intégration du contrôleur d'interruption sur une plateforme SoC.
Cet exercice ayant pour but de démontrer notre capacité à prendre du recul sur le module E3 commence par placer l'IP sur la plateforme FPGA.
Une description interne de l'IP compatible AXI4-Lite est ensuite proposée.
Sans rentrer dans les détails techniques, trois phases de tests sont exposées pour valider l'intégration.
Finalement, une partie axée sur la gestion du projet comprenant un planning et l'évaluation des risques est exposée.

Ce travail à la frontière entre conception technique et gestion de projet, a mis en lumière de nouvelles problématiques auxquelles nous n'avions pas l'habitude de répondre.
L'estimation temporelle à travers la réalisation du planning et l'affectation de durée aux tâches ont été délicates.
Ce travail d'intégration pourrait être poursuivie par une phase d'optimisation.
Elle viserait à réétudier la répartition matérielle/logicielle du SoC ainsi que la cible choisie (peut-être surdimensionnée).
