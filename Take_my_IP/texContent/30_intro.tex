\section{Introduction}

La formation ETN (Électronique et technologies numériques) option SETR (Systèmes embarqués temps-réel) offerte par l'école polytechnique de l'Université de Nantes propose d'aborder diverses branches de l'électronique, des systèmes temps-réel aux systèmes à microprocesseur en passant par la conception de SoC (système sur puce). 
Cet ensemble de domaines techniques nécessite des compétences en matière de méthodologie de conception. 
Ce rapport détaille la planification de l'intégration du contrôleur d'interruptions conçu dans le module E2.
Il est question de mettre l'accent sur les compétences transverses et sur notre capacité à prendre du recul sur les notions acquises cette année.
Il ne s'agit pas de faire le travail d'intégration, mais bien de le planifier, d'évaluer la charge de travail et anticiper les difficultés rencontrées.
Le scénario suivant est donné et guide ce rapport.

\gap

\textit{ Vous êtes le chef d'un projet dont le résultat serait un SOC opérationnel mis en œuvre au sein d’un FPGA. 
Ce SOC repose sur un processeur Zynq-PS. 
La plate-forme devra intégrer à terme l'IP que vous avez conçue dans le cadre du module E2. 
Vous devez présenter à votre responsable de service un dossier succinct mais convaincant du travail à mener afin d’effectuer l’intégration de votre IP au sein de la plate-forme. 
Il s’agit de définir l'effort nécessaire pour aboutir à ce résultat, sachant que la solution fonctionnelle de l'IP existe en VHDL synthétisable car vous l'avez conçue, testée et validée.}

\gap

Le travail présenté se positionne donc comme la suite logique du module E2.
Il peut être intéressant de rappeler sur le cycle en V l'avancement du projet.
\begin{figure}[H]
    \centering
    \includegraphics[width=0.75\linewidth]{cycle_v.draw.png}
    \caption{Positionnement de l'intégration sur le cycle en V}
    \label{fig:cycle_v}
\end{figure}
Sur ce cycle en V colorisé, représente en vert les étapes déjà réalisées et en rouge celle devant l'être pour finaliser le projet.
Parmi les étapes manquantes, le scénario proposé n'évoque que l'intégration.
Le test opérationnel sera donc laissé de côté.
Le  cycle en V donne une vue globale où on visualise bien l'étape précédente et le résultat souhaité.



