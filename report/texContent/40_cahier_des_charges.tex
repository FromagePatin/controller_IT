\section{Cahier des charges}
Cette partie présente le cahier des charges du périphériques.
Elle intègre les quelques points fournis par le sujet auquel s'ajoutent les contraintes imaginées par les étudiants.
\hl{...}
\subsection{Objectif du circuit}
Le contrôleur d'interruptions a l pour rôle d'informer le processeur sur l’occurrence d'une interruption valide.
Il fournira alors l’adresse de la prochaine instruction à exécuter.

\subsection{Fonctionnalitées attendues}
Les fonctions de service du circuit sont :
\begin{enumerate}
    \item Masquer et démasquer chaque interruption individuellement
    \item Contenir le vecteur d'exception
    \item Etablir le niveau de priorité des interruptions
    \item Ne fournir au \gls{CPU} que les interruptions valides
    \item Prendre en compte les priorités dans la génération des demandes au \gls{CPU}
\end{enumerate}
\subsection{Utilisation du circuit}

\subsection{Chronogrammes caractéristiques}

\subsection{Contraintes du projet}

