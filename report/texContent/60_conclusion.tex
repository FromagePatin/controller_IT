\section{Conclusion}


Pour conclure, la conception de l'IP contrôleur d'interruptions a été réalisée à l'aide de la méthode MCSE.
Il s'agissait de rappeler le cahier des charges, de présenter les spécifications et la conception.
Le principale avantage est d’effectuer une bonne structuration de sa conception, mais également de raisonner en faisant abstraction de la technologie utilisée. 
Cela permet de ne pas se confiner dans une solution technique. 
Également, la partie spécification est très utile pour dégrossir le cahier des charges ou pour réduire la complexité de conceptions lourdes, d’éviter les erreurs.
La conception de SoC est basée sur la conception hiérarchique.
L'IP est composée de blocs et de sous-blocs chacun vérifiés et validés individuellement puis en commun.
