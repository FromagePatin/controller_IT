\section{Spécifications}

\subsection{Caractérisation de l'environnement}

Il s'agit dans cette partie de caractériser l'environnement c'est-à-dire de d'identifier les entités interagissant avec le circuit à concevoir et décrire leur évolution à l'aide d'automates.
L'environnement du circuit à concevoir est constitué de trois entités :

\begin{itemize}
	\item Le processeur
	\item Le contrôleur mémoire fournissant le signal de sélection nCS\_IT.
	\item Les 15 entités informant au contrôleur d'interruptions la présence d'une interruption.
	Ce groupe d'entités est appelé "source d'interruptions".
\end{itemize}

L'ensemble processeur + contrôleur mémoire peut configurer le contrôleur d'interruptions et opérer des écritures et lectures au sein de ses registres. L'entité source d'interruptions envoie au circuit à concevoir la présence d'une interruption à traiter.


\subsection{Entrées et sorties du composant}