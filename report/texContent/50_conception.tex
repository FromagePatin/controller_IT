\section{Conception}

\subsection{Délimitation fonctionnelle}

La section suivante est un rappel de la délimitation du circuit à concevoir. Celle-ci présente le contrôleur d'interruptions d'un point de vue extérieur avec les 6 relations définies précédemment.

\begin{figure}[H]
	\centering
	\includegraphics[width=0.6\linewidth]{figure/delimitation_fonctionnelle.png}
	\caption{Délimitation fonctionnelle du contrôleur d'interruptions}
	\label{fig:delimitation_fonctionnelle}
\end{figure}

Il s'agira par la suite de se placer dans un point de vue interne afin de présenter les différents blocs composant le contrôleur d'interruptions avec ses relations internes.
L'évènement \texttt{Ack\_Read} permet d'acquitter la lecture de l'adresse de branchement par le processeur. Cette relation peut être déduite selon le contenu du bus d'adresse (si la lecture est faite dans le registre \texttt{blx} alors il y a acquittement).  

\subsection{Décomposition fonctionnelle et introduction des interfaces}

La figure ci-dessous présente le circuit interne observé de l'intérieur avec l'introduction des signaux physiques du circuit. Il existe deux blocs nommés \texttt{Interface} et \texttt{Traitement}. 

\begin{figure}[H]
	\centering
	\includegraphics[width=1\linewidth]{figure/decomposition_fonctionnelle.png}
	\caption{Décomposition fonctionnelle et introduction des interfaces}
	\label{fig:decomposition_fonctionnelle}
\end{figure}

\newpage

Le bloc \texttt{Traitement} est chargé de signaler le CPU d'une interruption à traiter à la suite de multiples opérations (traitement des priorités, opération de masquage, mise en suspens).
Le bloc \texttt{Interface} permet de faire la jonction entre les signaux de contrôle et bus du système à microprocesseur et l'IP à concevoir.
\texttt{Ack\_Read} est en sortie du bloc \texttt{Interface} et avertit le bloc \texttt{Traitement} de la lecture de l'adresse de branchement.
Il ne s'agit plus d'un événement mais d'une relation.
C'est une variable partagée qui engendrera une action sur niveau et non sur front.

\subsection{Raffinage du bloc \texttt{Interface}}

La section suivante présente un raffinage, c'est-à-dire de réaliser une seconde décomposition en se plaçant à l'intérieur du bloc \texttt{Interface}.
Ce bloc est composé de \texttt{Interface\_Write} et \texttt{Interface\_Read}. 

\begin{figure}[H]
	\centering
	\includegraphics[width=1\linewidth]{figure/raffinage_interface.png}
	\caption{Raffinage et décomposition du bloc \texttt{Interface}}
	\label{fig:raffinage_interface}
\end{figure}

Cette décomposition en bloc hiérarchique permet de mettre en lumière deux fonctions du bloc \texttt{Interface}.
Celui-ci peut écrire dans les registres du contrôleur d'interruptions par le biais du bloc \texttt{Interface\_Write} et les lire avec \texttt{Interface\_Read}.

\newpage

\subsection{Raffinage du bloc \texttt{Traitement}}

Après le raffinage du bloc \texttt{Interface}, il s'agit désormais de décrire le point de vue interne du bloc \texttt{Traitement}.
La structure est donnée ci-dessous.

\begin{figure}[H]
	\centering
	\includegraphics[width=1\linewidth]{figure/raffinage_traitement.png}
	\caption{Raffinage et décomposition du bloc \texttt{Traitement}}
	\label{fig:raffinage_traitement}
\end{figure}

Ce bloc présente trois éléments appelés respectivement \texttt{Gestion adresse de branchement}, \texttt{Solutionneur de priorité} et \texttt{Masquage}.
Le bloc \texttt{Masquage} utilise les 15 interruptions ainsi que le registre masque pour filtrer les interruptions que l'utilisateur souhaite traiter.
Le sous-bloc \texttt{Solutionneur de priorité} est utilisé pour traiter les interruptions selon leur niveau de priorité et d'indiquer l'identifiant de l'interruption à traiter.
Le signal \texttt{Is\_prio\_active} informe au bloc \texttt{Gestion adresse de branchement} si l'interruption en cours de traitement est toujours active, c'est-à-dire à l'état bas.
Le bloc \texttt{Gestion adresse de branchement} récupère l'identifiant de l'IT correspondant à l'indice du registre @handler contenant l'adresse de branchement à placer dans @blx.
Pour que le processeur puisse lire l'adresse de branchement le signal nIT\_CPU est activé.
Suite à la lecture de la part du processeur le signal Ack\_Read est actif afin d'informer le contrôleur d'interruption la fin de la phase de lecture et remettre nIT\_CPU à l'état logique '1'.  

\newpage

\subsection{Écriture des algorithmes}
\subsubsection{Algorithme \texttt{Interface\_Write}}

L'algorithme ci-dessous représente le comportement de \texttt{Interface\_Write}.
Ce bloc permet l'écriture synchrone par le processeur dans les registres internes du contrôleur d'interruption. 

\begin{lstlisting}[style=pascalstyle]
action Interface_Write sur clk avec
(
entree var d_bus 			: Def_data;
entree var addr 			: Def_addr;
entree var nRST 			: Def_Bit
entree var nCS_IT 			: Def_Bit;
entree var nAS 				: Def_Bit;
entree var RnW 				: Def_Bit;
sortie var EN 				: Def_Bit;
sortie var vect_priorite 	: Def_vect_priorite;
sortie var vect_handler 	: Def_vect_handler;
sortie var masque 			: Def_masque
);

const addr_EN 				: Def_addr = 0x000000;
const addr_masque 			: Def_addr = 0x000002;
const addr_vect_handler 	: Def_addr = 0x00000A;
const addr_vect_priorite 	: Def_addr = 0x000044;

begin

cycle:
begin
	cycle H:
	begin
		if (nRST=0) then
			EN := 0;
			priorite := 0;
			vect_handler := 0;
			masque := 0;
		end if;
		if (nCS_IT=0) et (RnW=0) et (nAS=0) then
			case addr of
			addr_EN:
				EN := d_bus;
			addr_masque:
				masque := d_bus;
			addr_vect_handler:
				vect_handler := d_bus;
			addr_vect_priorite:
				vect_priorite := d_bus;
		end if;
	end cycle H;
end cycle;
end Interface_Write;
\end{lstlisting}

\newpage
\subsubsection{Algorithme \texttt{Interface\_Read}}

L'algorithme ci-dessous représente le comportement de \texttt{Interface\_Read}.
Ce bloc permet la lecture asynchrone par le processeur dans les registres internes du contrôleur d'interruption.
Un cas particulier est à considérer lors de la lecture de l'adresse de branchement blx.
Le signal Ack\_Read passe à l'état logique 1 pour un cycle horloge afin de signaler au bloc \texttt{Traitement} la lecture de l'adresse de branchement par le processeur.
Ack\_Read est utilisé comme condition de transition de la machine état \texttt{Gestion adresse de branchement}.

\begin{lstlisting}[style=pascalstyle]
action Interface_Read sur message nRST, EN, vect_priorite, vect_handle,
masque, pending, blx avec
(
entree var addr 			: Def_addr;
entree var nCS_IT 			: Def_Bit;
entree var nAS 				: Def_Bit;
entree var RnW 				: Def_Bit;
entree var nRST 			: Def_Bit;
entree var EN 				: Def_Bit;
entree var vect_priorite 	: Def_vect_priorite;
entree var vect_handler 	: Def_vect_handler;
entree var masque 			: Def_masque;
entree var pending 			: Def_pending
entree var blx 				: Def_blx;
sortie var d_bus 			: Def_data;
sortie var Ack_Read			: Def_bit;
);

var TriState 				: HauteImpedance;
const addr_EN	     		: Def_addr = 0x000000;
const addr_masque   		: Def_addr = 0x000002;
const addr_pending 			: Def_addr = 0x000004;
const addr_blx 				: Def_addr = 0x000006;
const addr_vect_handler 	: Def_addr = 0x00000A;
const addr_vect_priorite 	: Def_addr = 0x000044;

begin

cycle:
begin
	if (nRST=0) then
		Ack_Read := 0;
	end if;
	d_bus := TriState;
	Ack_Read := 0;
	if (nCS_IT=0) et (RnW=1) et (nAS=0) then
		case addr of
		addr_EN:
			d_bus := EN;
		addr_masque:
			d_bus := masque;
		addr_vect_handler:
			d_bus := vect_handler;
		addr_vect_priorite:
			d_bus := vect_priorite;
		addr_pending:
			d_bus := pending;
		addr_blx:
			d_bus := blx;
			Ack_Read := 1;
	end if;
end cycle;
end Interface_Read;
\end{lstlisting}

\newpage

\subsubsection{Algorithme \texttt{Masquage}}

L'algorithme ci-dessous est celui du bloc \texttt{Masquage}.
Le langage utilisé est également le Pascal.
Le type \texttt{Def\_IT\_xxx} est un entier de 0 à 14. 

\begin{lstlisting}[style=pascalstyle]
action Masquage sur message nIT_xxx avec
(
entree var nIT_xxx 			: Def_IT_xxx;
entree var mask 			: Def_masque;
sortie var nIT_xxx_masked 	: Def_IT_xxx;
);

begin

cycle:
begin
	nIT_xxx_masked := nIT_xxx and not mask;
end cycle;
end Masquage;
\end{lstlisting}

\subsubsection{Algorithme \texttt{Solutionneur de priorité}}

Il s'agit ensuite de l'algorithme représentant le comportement du sous-bloc \texttt{Solutionneur de priorité}.
Lors du passage à l'état logique 1 d'une des interruptions, la priorité correspondante à cette IT est évaluée avec la priorité max (c'est-à-dire la priorité de l'interruption en cours).
Si la priorité de la \texttt{i}-ème IT est supérieure à la priorité de l'IT en cours, celle-ci passe comme IT à traiter.

\begin{lstlisting}[style=pascalstyle]
action Solutionneur_de_priorite sur message nIT_xxx avec
(
entree var nIT_xxx 				: Def_IT_xxx;
entree var priority_vector 		: Def_vect_priorite;
sortie var ID_IT 				: Def_ID;
);

var max_prio 					: integer = 0;
var temp_ID_IT 					: Def_ID = 0;
var IT_size 					: integer = 15;
var ID 							: Def_ID;

begin

cycle:
begin
	for i := 0 to IT_size-1 do
	begin
		if nIT_xxx(i) = 1 then
			if priority_vector(i) > max_prio then
				max_prio := priority_vector(i);
				temp_ID_IT := ID(i);
			else if (priority_vector(i)=max_prio) and (i>temp_ID_IT) then
				temp_ID_IT := ID(i);
			end if;	
		end if;
	end for;

	ID_IT := temp_ID_IT;

end cycle;
end Solutionneur_de_priorite;
\end{lstlisting}

\newpage

\subsubsection{Algorithme \texttt{Gestion adresse de branchement}}

Pour finir il reste l'algorithme de \texttt{Gestion adresse de branchement}.
Il s'agit d'une machine état séquentielle finie chargée de signaler le CPU d'une IT à traiter, placer l'adresse de la routine d'interruption dans le registre blx et attendre la lecture.
Il est possible de remarquer la structure typique d'une machine à état finie.
Il y a le bloc de sortie de la machine à état finie qui est combinatoire puis le bloc calculant l'état suivant synchrone à l'horloge du système.   



\begin{lstlisting}[style=pascalstyle]
action Gestion_blx sur clk, message nRST, message state avec
(
entree var clk 				: Def_bit;
entree var nRST 			: Def_bit;
entree var ID_IT 			: Def_ID;
entree var Ack_Read 		: Def_bit;
entree var Is_IT_active 	: Def_bit;
entree var vect_handler 	: Def_vect_handler;
sortie var nIT_CPU 			: Def_bit;
sortie var blx 				: Def_addr;
);

var state 					: Def_state;
var prev_ID_IT 				: Def_ID;

begin

cycle:
begin
	if nRST = 0 then
		state := WAIT_IT;
		prev_ID_IT := 0;
		nIT_CPU := 1;
		blx := 0;
	else
		case state of
		WAIT_READ :
			prev_ID_IT := ID_IT;
			nIT_CPU := 0;
			blx := vect_handler(ID_IT);
		NEXT_IT :
			nIT_CPU := 1;
	end if;
	cycle H:
	begin
		case state of
		WAIT_IT :
			if is_IT_active = 1 then
				state := WAIT_READ;
			end if;
		WAIT_READ :
			if Ack_Read = 1 then
				state := NEXT_IT;
			end if;
		NEXT_IT :
			if prev_ID_IT <> ID_IT then
				state := WAIT_READ;
			else if 
				state := WAIT_IT;
			end if;
	end cycle H;
end cycle;
end Gestion_blx;
\end{lstlisting}